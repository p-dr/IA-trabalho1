\documentclass{article}
\usepackage[utf8]{inputenc}
\usepackage[margin=2cm]{geometry}

\begin{document}
\section{Criação dos tabuleiros}
Criou-se algoritmo para a geração de tabuleiros, com suas paredes e os pontos de início e fim da busca. Esse processo é executado pelo script \texttt{gen\_boards.py}, e acontece da seguinte maneira:

\begin{enumerate}
	\item É criado um tabuleiro (matrix) em branco (preenchido com zeros), a partir das dimensões informadas com argumento;
	\item São sorteados dois pontos aleatoriamente para servirem de início e fim da busca. São sorteadas quantas vezes forem preciso até que tenham distância Manhattan entre si maior que um comprimento heuristicamente definido como a soma das dimensões do tabuleiro sobre 2;
	\item São construídas as paredes, como melhor explicado posteriormente.
\end{enumerate}

Inicialmente criava-se um caminho po
\end{document}
